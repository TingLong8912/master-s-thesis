\begin{table}[htbp]
\centering
\caption{位置描述範例}
\label{tab:dailyLocaTable}
\begin{adjustbox}{max width=\textwidth}
\renewcommand{\arraystretch}{1.4}
\begin{tabular}{
>{\centering\arraybackslash}m{1cm} >{\centering\arraybackslash}m{3.5cm} >{\centering\arraybackslash}m{7.5cm} >{\centering\arraybackslash}m{5cm}}
\Xhline{1.2pt}
項次 & 階段 & 規則簡單範例 & 敘述 \\
\Xhline{1.2pt}
1 & 情境影響參考物選擇 & \begin{lstlisting}[language=Prolog, basicstyle=\ttfamily, xleftmargin=2em]
AnyContext(?ctx)^AnyType(?ref)^SpatialOpeartion(?rel) 
->GroundFeature(?ref)^hasGroundFeature(?rel, ?ref)
\end{lstlisting} & 任一情境下選擇任一物件類型作為參考物 \\
\hline
2 & 資料層到語意層 & \begin{lstlisting}[language=Prolog, basicstyle=\ttfamily, xleftmargin=2em]
AnyContext (?ctx)^Distance(?rel)
^hasDistance(?rel, ?d)^swrlb:lessThan(?d, 300)
->Near(?rel)
\end{lstlisting} & 在任一情境中少於300被視爲「近」 \\
\hline
3 & 語意層到自然語言層 & \begin{lstlisting}[language=Prolog, basicstyle=\ttfamily, xleftmargin=2em]
WorsSense(?s)^Word(?w)
->hasSpatialPreposition(?w)/hasLocaliser(?w)
\end{lstlisting} & 詞意詞彙對照 \\
\hline
\Xhline{1.2pt}
\end{tabular}
\end{adjustbox}
\end{table}