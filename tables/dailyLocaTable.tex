\begin{table}[htbp]
\centering
\caption{位置描述範例}
\label{tab:dailyLocaTable}
\begin{adjustbox}{max width=\textwidth}
\renewcommand{\arraystretch}{1.4}
\begin{tabular}{>{\centering\arraybackslash}m{1cm} >{\centering\arraybackslash}m{7.5cm} >{\centering\arraybackslash}m{3.5cm} >{\centering\arraybackslash}m{5cm}}
\Xhline{1.2pt}
項次 & 內容 & 類型 & 來源 \\
\Xhline{1.2pt}
1 & 時間:2025/01/23 09:51,位置:北部(國道3號北上46km),內容:北上在46公里,樹林交流道,小貨???-6712未保持安全距離 & 即時路況:其他 & 警察廣播電台 https://rtr.pbs.gov.tw/pbsmgt/RoadAll.html \\
\hline
2 & 時間:2025/01/23 08:42,位置:北部,內容:新生南路+和平東路2段口,路面改善工程,施工至114/2/16,每日0800$\sim$1700 & 即時路況:道路施工 & 警察廣播電台 https://rtr.pbs.gov.tw/pbsmgt/RoadAll.html \\
\hline
3 & [地震速報 Earthquake Alert] 01/30 15:16 左右南部地區發生顯著有感地震,慎防強烈搖晃,就近避難「趴下、掩護、穩住」,氣象署。Felt earthquake alert. Keep calm and seek cover nearby. CWA 02\_2349 1181 避難宣導:https://gov.tw/KNs & 災防告警:地震速報 & 災防告警細胞廣播訊息 https://cbs.tw/25013cdcf3a4 \\
\hline
4 & [水壩放水警戒] 台電木瓜壩將於03月24日09時30分起排砂放水至木瓜溪,請木瓜溪下游沿岸民眾加強注意並遠離河床,以策安全。Reservoir discharge 09:30 MUGUA River level rise. Plz stay away. 1911 TAIPOWER (台電公司1911). Taipower03-8350161 & 災防告警:水庫放水警戒 & 災防告警細胞廣播訊息 https://cbs.tw/25033cdcf541 \\
\Xhline{1.2pt}
\end{tabular}
\end{adjustbox}
\end{table}