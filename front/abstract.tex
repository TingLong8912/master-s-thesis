% !TeX root = ../main.tex

\begin{abstract}

位置描述(Location description)係人們在日常對話中位置與地點的自然語言表達,其指稱地址或地名等專有名詞,或在與空間參考物產生關係下,由人類空間認知理解環境後所指涉的絕對或相對位置。然而,在地理資訊系統(Geographic Information System, GIS)中多以幾何坐標或屬性資料儲存和呈現位置,較少觸及位置概念的表達,亦未充分考量使用者的空間認知與情境解讀。

本研究旨在拓展基於坐標之空間資料與空間關係的位置與地點表達,著重在結合知識本體進行建立空間語意,並藉由語意規則與推論機制產生自然語言文字,使得所查詢之各類圖徵皆可具有適切之位置描述。研究方法運用知識本體建構空間資料中物件性質與空間關係的形式化(formalization)語意表徵結果,以量化、推論與產生不同圖徵類型與個人位置與地點之描述,使得圖徵位置得以依據不同空間尺度與情境進行語意調整與表達。

在案例研究上,本研究選擇交通及災防情境作為驗證本研究機制的可行性與適切性,提供實際應用於導航、道路救災、即時路況廣播或災防告警細胞廣播訊息等場域。本研究貢獻在於提升在空間資料中對於各種尺度與考量情境的位置語意表達,並建立結合GIS進行空間分析與語意推論的整合性框架,期拓展人們使用空間資料和解讀空間資訊之有效途徑並產生實務應用價值。


\end{abstract}

\begin{abstract*}

A location description denotes the use of natural language to express spatial positions in human communication. It encompasses not only proper nouns (such as addresses or place names), but also spatial concepts derived from spatial cognition related to reference objects. However, in computational environments such as Geographic Information Systems (GIS), spatial data and their relationships are primarily represented as geographically referenced values, i.e., coordinates enriched metadata (attributes), which often lack flexible semantic representations for contextual interpretation between spatial objects. 

This study addresses this gap by integrating ontologies to construct spatial semantics and generate location descriptions through semantic inference rules. The proposed framework formalizes spatial features and relationships embedded in coordinate-based and multi-dimensional spatial data, enabling the quantification, inference, and generation of context-aware descriptions across different spatial scales and levels of prominence.

In this case study, the traffic and disaster prevention scenario is selected to verify the feasibility and applicability of the proposed mechanism. The framework is designed to support real-world applications such as navigation, emergency road response, real-time traffic broadcasting, and disaster alert cell broadcast messaging.

This study contributes to enhancing the semantic representation of location by considering various spatial scales and contextual factors within spatial data. Furthermore, it establishes an integrated framework that combines GIS-based spatial analysis with semantic reasoning, aiming to expand effective ways for users to interpret and utilize spatial information and thereby generate practical value.


\end{abstract*}