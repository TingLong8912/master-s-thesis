% !TeX root = ../main.tex

\chapter{緒論}

位置(Location)意指「所在的地方」\citep{RN171},或可定義為「 a position or site occupied or available for occupancy or marked by some distinguishing feature」\citep{RN172},譯為可被佔據的一空間位置、地點,或具有特徵可供識別的區域。進一步,在表達某地的位置上,可分為絕對位置及相對位置\citep{RN171}。絕對位置通常以特定空間參考系統(spatial reference system)為基準,常見為經緯度,例如:台北101位於北緯25.034026度、東經121.563903度\citep{RN174};相對位置則是基於距離和方位等空間關係作為基準表達,描述各物理存在之空間物件(spatial objects)之間的相對空間關係,例如:「台北101緊鄰台北市政府」中空間物件為物理存在地表可見的台北101和台北市政府,其中台北市政府為參考物,並以距離表達與目標物台北101的關係。

在人類的空間認知與日常溝通中,位置不僅僅是一個靜態的概念,還涉及如何理解與表徵它。其中,位置描述(Location description)是人類將自身空間知識透過自然語言表徵(representation)進行轉譯和編碼,用以描述位置,以支持日常溝通位置之所在的途徑\citep{RN23, RN93, RN126, RN128}。自然語言式的位置描述是人類進行溝通的重要方式,能以直觀的方式傳遞空間資訊,使得接收者快速掌握所在位置。位置描述廣泛存在於日常生活與專業領域中,例如:日常對話、社群媒體貼文、新聞報導,甚或是專業領域中的文書報告、即時路況或災防告警細胞廣播訊息等。無論是表達位置座落、事件發生地點,或用以告知所在地,其核心目的皆為回答「所在位置」的問題。常見的位置描述包含,地名、地標、門牌位置和相對位置描述等,例如:台北101(參考系統為地標)、台北市信義區市府路45號(參考系統為門牌位置),又如象山位於台北盆地東緣(參考物為台北盆地,並以方位關係表達)。

\section{研究背景}

自然語言式的位置描述是人類進行溝通的重要方式,能以直觀的方式傳遞空間資訊,不僅廣泛應用於各種情境,也存在依情境而呈現多樣化的空間參考物件選擇、語法結構與空間尺度之細緻度,例如:在日常導航情境中,表達位置描述常會依照路段名稱或地標等作為參考物。表 1展示在實際情境(包含即時路況資訊和災防細胞廣播訊息)中的位置描述範例,以警廣發布之位置描述為例(項次1和2),舉凡是災變、道路施工,或是其他大大小小可能影響行車的路況資訊皆有傳達位置所在的必要性,所產生的位置描述之參考物也涵蓋各級道路,包含:國道、省道或市區道路等。同時,為使訊息接收者清晰認知位置,除道路對象外,會包含其他地標或附屬設施等參考物,例如:項次1中的地理區劃—「北部」、方向—「北上」、道路附屬設施—「樹林交流道」等資訊;而在更為細緻的道路分級中(項次2),包含的參考物為路名—「新生南路」和「和平東路2段口」。

同樣地,國家災害防救科技中心在災防告警上發布的災害告警細胞廣播訊息,為說明災防事件或影響地點也存在位置描述(項次3和4),當中也能看到位置描述是由多組參考物及其空間關係所構成,會以不同情境選擇不同類別的參考物;而在同個情境下隨著空間尺度不同又會以不同類別之參考物輔助描述,以增進訊息接收者認知事件或影響地點所在。如在地震速報一類,中會放上「南部地區」等描述(項次3),在水庫放水警戒中會提醒事發地「木瓜壩」及影響範圍「木瓜溪下游沿岸附近」(項次4)。

\begin{table}[htbp]
\centering
\caption{位置描述範例}
\label{tab:dailyLocaTable}
\begin{adjustbox}{max width=\textwidth}
\renewcommand{\arraystretch}{1.4}
\begin{tabular}{>{\centering\arraybackslash}m{1cm} >{\centering\arraybackslash}m{7.5cm} >{\centering\arraybackslash}m{3.5cm} >{\centering\arraybackslash}m{5cm}}
\toprule
項次 & 內容 & 類型 & 來源 \\
\toprule
1 & 時間:2025/01/23 09:51,位置:北部(國道3號北上46km),內容:北上在46公里,樹林交流道,小貨???-6712未保持安全距離 & 即時路況:其他 & 警察廣播電台 https://rtr.pbs.gov.tw/pbsmgt/RoadAll.html \\
\hline
2 & 時間:2025/01/23 08:42,位置:北部,內容:新生南路+和平東路2段口,路面改善工程,施工至114/2/16,每日0800$\sim$1700 & 即時路況:道路施工 & 警察廣播電台 https://rtr.pbs.gov.tw/pbsmgt/RoadAll.html \\
\hline
3 & [地震速報 Earthquake Alert] 01/30 15:16 左右南部地區發生顯著有感地震,慎防強烈搖晃,就近避難「趴下、掩護、穩住」,氣象署。Felt earthquake alert. Keep calm and seek cover nearby. CWA 02\_2349 1181 避難宣導:https://gov.tw/KNs & 災防告警:地震速報 & 災防告警細胞廣播訊息 https://cbs.tw/25013cdcf3a4 \\
\hline
4 & [水壩放水警戒] 台電木瓜壩將於03月24日09時30分起排砂放水至木瓜溪,請木瓜溪下游沿岸民眾加強注意並遠離河床,以策安全。Reservoir discharge 09:30 MUGUA River level rise. Plz stay away. 1911 TAIPOWER (台電公司1911). Taipower03-8350161 & 災防告警:水庫放水警戒 & 災防告警細胞廣播訊息 https://cbs.tw/25033cdcf541 \\
\bottomrule
\end{tabular}
\end{adjustbox}
\end{table}

然而,當位置資訊被形式化(formalization)處理與分析時,則需轉換成結構化和機器可處理的空間表示形式。相較於人類日常溝通中的靈活且具情境依賴的自然語言式位置描述,目前地理資訊系統(Geographic Information System, GIS)則傾向以空間坐標所構成的圖徵(feature)及其屬性(attribute)來靜態呈現空間物件\citep{RN126},例如:POINT(121.563903, 25.034026)及名稱屬性為「台北101」。而在空間資訊領域中,為使位置描述可動態產生,地理編碼(Geocoding)和逆向地理編碼(Reverse Geocoding)提供位置描述和GIS記錄之空間物件轉換處理方式,提供了非地理領域專家及一般民眾略過讀圖和操作GIS的階段而快速解讀位置所在。

地理編碼係為將自由文字描述轉換為空間坐標值表示,利於計算機進行資料運算與空間分析;而逆向地理編碼則將坐標值轉換為自由文字描述,轉換結果可為地址(含縣市、鄉鎮市區、路名)或地名等\citep{RN18, RN17}。儘管現有逆向地理編碼技術提供了以空間資料查詢與轉換成自然語言式的位置描述,但其仍多侷限於靜態、階層化與地址導向的資訊呈現,難以反映人類日常語言中靈活且具語意的位置描述\citep{RN122},例如,無法依據語境回應「在公館捷運站附近」、「靠近大安森林公園」等具體且相對關係強的描述方式。

舉例而言,表 2顯示常見的開源和商業逆向地理編碼服務或應用程式介面(Application Programming Interface, API)範例,包含Google Reverse Geocoding \citep{RN175}、OSM Nominatim \citep{RN176}、ArcGIS reverseGeocode \citep{RN177}、Mapbox Reverse geocoding \citep{RN178}、 Bing reverse geocode \citep{RN179}和GeoNames \citep{RN180}等,透過輸入查詢點坐標(台北101的經緯度坐標),所獲得的各種結果。綜述來說,目前的逆向地理編碼機制以提供「address」資訊為主,又當無合適之地址,會自動以較高層級行政區資訊回傳,鮮少考量如何傳遞位置描述使使用者更能認知其所在,唯少數如ArcGIS和GeoNames提供部份POI資訊,然其仍限於給予使用者靜態和階層化的資訊。

\begin{table}[htbp]
\centering
\caption{逆向地理編碼範例}
\label{tab:reverseGeocodingAPI}
\begin{adjustbox}{max width=\textwidth}
\renewcommand{\arraystretch}{1.4}
\begin{tabular}{>{\centering\arraybackslash}m{3cm} >{\centering\arraybackslash}m{7.5cm} >{\centering\arraybackslash}m{2.5cm} >{\centering\arraybackslash}m{5cm}}
\Xhline{1.2pt} 
使用服務 & 結果 & 等級 & 備註 \\
\Xhline{1.2pt}
Google Reverse Geocoding & No. 45 號, Section 5, Xinyi Rd, Xinyi District, Taipei City, Taiwan 110 & address & 台北101門牌號碼為7號,故定位精準度不甚準確 \\
\hline
OSM Nominatim & 7 號, Section 5, Xinyi Road, Xicun Village, Xinyi District, Xinyi Commercial Zone, Taipei, 11049, Taiwan & address & 可進一步從OSM物件查詢結果中取得「Building」等其他 tag 資訊 \\
\hline
ArcGIS REST APIs reverseGeocode & 台 北 市 信 義 區 信 義 路五段 7 號 台北德式診所 信義路五段 & address & 可依照需求選擇參數,例如:address或POI,獲得結果 \\
\Xhline{1.2pt} 
\end{tabular}
\end{adjustbox}
\end{table}

\section{研究動機與問題}

隨著定位技術的普及與行動裝置的廣泛使用,目前產生大量以坐標為基礎的空間資料。例如,全球定位系統(global positioning system, GPS)提供了民眾快速記錄下空間位置的能力,同時帶有地理標記(geotagged)的社群媒體貼文或照片大量存在網路世界中。然而,相較於可提供人們快速解讀位置的自然語言式位置描述,這些蘊含豐富空間資訊的坐標資料往往缺少可讀性。逆向地理編碼技術雖能將坐標轉換為地址或地名,目前在透過坐標為基礎的空間資料表達文字式之位置描述,當中涉及的流程與語意處理,仍面臨以下四項關鍵挑戰,顯示其在語意理解與形式化表達的能力上之不足:

\begin{enumerate}
\item 位置描述的人工撰寫難以解讀且缺乏一致性:現行自然語言位置描述多仰賴人工撰寫,導致資訊片段、內容不完整,特別在災防告警等需即時傳遞訊息的情境中,語意模糊與表達不一致的問題將影響民眾的理解與反應效率。
    \item 逆向地理編碼輸入與輸出受限,無法涵蓋多樣位置描述需求:目前逆向地理編碼機制多僅支持單一坐標點輸入與地址類輸出,難以處理線、面、多重幾何圖徵等實務需求,也無法產出除門牌地址外的資訊以達成多樣化語意描述。
    \item 情境對位置描述具有高度影響力:位置描述所依據的參考物與表達方式,會隨不同使用情境而異,諸如災防告警與道路路況需參照的空間物件類型與描述語彙皆有所不同,然現行機制未能納入情境語意。
    \item 缺乏對空間物件與其關係的語意處理:目前逆向地理編碼多僅以最短距離為依據選擇參考物,未考慮語意適切性,易產生違反認知預期的結果,忽略使用者與參考物之間的語意與認知。
\end{enumerate}

綜上所述,隨著災防管理、社群媒體與基於位置的服務的發展,對於具語意的自然語言位置描述需求日益增加。然而,目前在GIS中對語意資訊的處理能力仍侷限於幾何層級,缺乏支持語意理解位置與產生位置描述的方法。因此,賦予空間資料在位置上的語意理解與表達能力,成為需突破的核心課題。

知識本體源於哲學領域,係為旨在研究物體存在(being)之科學,特別關注事物的本質和結構,在電腦科學領域則採借這一概念,用來探討物件領域的類型、結構、屬性、過程及其相互關係\citep{RN160, RN92}。本研究提出以知識本體作為語意建模與推導的核心方法,建構一套可於GIS操作中自動化產出自然語言位置描述之語意框架。透過概念化與形式化的方式,知識本體有助於捕捉現實世界中抽象現象與具體物件之間的語意結構,補足目前在空間資訊中表達位置語意的不足。具體而言,本研究涵蓋研究問題為:

\begin{enumerate}
    \item 如何運用知識本體形式化建模語意結構,以整合空間資料、幾何計算上的空間關係、語意空間關係及空間認知的知識框架?
    \item 如何根據語意觀點從坐標為基礎的空間資料中產生位置描述,當中涵蓋情境與空間認知因素影響位置描述中的語意?
    \item 如何設計並實作一套可於GIS中操作的語意式位置描述,使其能根據使用者任意繪圖或輸入的多維幾何,自動產生具豐富性與情境依賴的自然語言式位置描述?
\end{enumerate}

\section{章節安排}

本研究共分為六個章節:第二章回顧相關文獻,說明各領域對於「位置」及「位置描述」的探討,並根據此作為建構本研究位置描述知識本體之理論基礎,同時指出現有研究中對於產生位置描述在研究方法尚待解決的機會與挑戰;第三章提出本研究核心架構「位置描述語意三層式架構」和當中「位置描述知識本體」的設計想法與內容,以及如何將本研究框架擴展至特定情境上;第四章則說明進一步分析如何將語意結構整合至GIS中,以支持從基於坐標的空間資料產生語意式位置描述;在第五章則說明本研究建構之知識本體與系統架構在實際案例中的結果討論與評估;最後,在第六章總結本研究主要貢獻與創新,並提出未來研究在語意推論、結合GIS的自動化應用與領域知識本體可延伸之研究方向。