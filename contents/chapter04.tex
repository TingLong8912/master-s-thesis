% !TeX root = ../main.tex

\chapter{基於知識本體之位置描述的網路網路地理資訊系統應用整合}

本研究開發了一個基於知識本體之語意式位置描述的網際網路地理資訊系統(Web Geographic Information System, WebGIS)應用,其核心架構整合知識本體、SWRL規則推論、GIS和網頁,以實現語意式位置描述的產生與應用(參見圖 11)。

本研究的系統架構包含兩大資源層,分別為「GIS資源層」和「語意資源層」。GIS資源層負責Feature記錄與空間操作,本研究採用PostgreSQL空間資料庫進行儲存,以提供API介接,並支援應用開發;語意資源層涵蓋位置描述知識本體與SWRL規則集,用以推導位置語意資訊,確保語意式位置描述的完整性與合理性。本研究在知識本體的建構過程中,採用OWL (Web Ontology Language)作為知識表示工具,並利用Protégé進行編輯,以確保語意結構的完整性與標準化。此外,本研究設計了一套基於SWRL的推論規則集,該規則可用於模擬空間物件的關係推理與語意合理性檢驗,當中包含兩階段的推論(即空間關係推論和詞意推論)。

本研究的WebGIS應用採用API介接模式,使用者發送請求並接收回應,其中:請求為任一Feature,回應則為文字式的位置描述。程式執行流程概述如下:第一,空間關係查詢與紀錄,即透過輸入Feature和存在於資料庫中的參考空間物件。本研究開發API提供空間關係查詢與記錄,開發環境選用為Node.js,並結合基於JavaScript的空間操作函式庫Turf.js進行方便且快速的空間關係查詢。而後,進行語意推論與位置描述生成。根據前述所獲得的空間操作關係組,會進一步將內容映射至知識本體,並透過SWRL規則進行推論。該一過程透過Python的Owlready2套件進行知識本體的載入、推理與規則驗證,以實踐自動化推理功能,當中的推論器為Pellet reasoner。

本研究最終將發展一組API及WebGIS應用,以坐標產生位置描述的逆向地理編碼機制,提升地理資訊在多元領域的應用價值。系統主要包含兩項功能:其一,使用者可透過API存取服務,將現有的歷史資料中附有坐標的空間資訊自動轉換為文字化的位置描述,滿足大規模資料處理需求;另一,透過WebGIS介面,使用者可以即時繪製空間範圍,並獲取範圍內對應的位置描述結果。該功能提供即時性且直覺的使用體驗,特別適用於探索性分析與小範圍資料需求。此外,為促進研究持續增進,在介面上將設計回饋機制以蒐集結果準確性及適宜性的評估資料,進一步作為未來改良位置描述編碼機制之參考基礎。
