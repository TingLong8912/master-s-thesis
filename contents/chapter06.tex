% !TeX root = ../main.tex

\chapter{結論與未來展望}

本研究不僅拓展了逆向地理編碼的新視角,也在交通與災防告警的情境中展現了具體的實務可能和貢獻。在研究方法上,本研究所提出之 O-SLD 位置描述生成機制具有以下三項主要特色:第一,可接受點、線、面等各種形狀的空間物件作為輸入;第二,可彈性結合多個空間參考物;第三,可整合知識本體與 SWRL 規則進行語意限制。這些特性使得產生之自然語言式位置描述具備更豐富的結構與資訊內容,從空間關係與尺度觀點更貼近人類理解的位置,實現將GIS坐標紀錄之物件轉譯為具語意更易溝通及閱讀的位置描述文字。

在實務應用層面,系統已於災防告警與交通案例中展現效益,能根據情境及空間尺度,自動化產生合適的位置描述,提供更貼近使用者認知的語句建議,進而應用於基於位置服務之地點描述、警示訊息自動生成與地圖輔助描述等,例如:災防單位結合本研究提出 O-SLD 機制,僅需輸入 GeoJSON,即可快速獲得多組語意化位置描述建議,提升作業效率與溝通清晰度。

未來研究可延伸本研究之架構,發展更多樣化的語意位置描述類型,並與特定領域情境結合,擴展知識本體與語意規則於自然語言地理描述的應用層面,進一步提升語意標註的廣度與深度。本研究亦存在若干限制。首先,位置描述的精確性仍依賴知識本體的完備程度與參考物的給予,當參考物資料不完整時,本研究機制無法從無資料中產生最理想的描述。例如,目前知識本體及空間物件僅考量二維GIS中的表達,涉及三維的地物則無法處理,仍待未來研究進一步結合本研究發展之位置描述知識本體擴充。其次,本研究結合知識本體建構出人如何理解位置的理論框架,然實際語意推論能力受制於情境及使用者的多元性,未來研究可透過經驗研究方式待量化人的空間感知與語句偏好,進一步提升位置描述的語意精準度和使用者體驗。
