% !TeX root = ../main.tex

\chapter{案例研究}

本章說明位置描述知識本體和O-SLD框架的可行性實踐。本研究選擇警察廣播電台即時路況與災防告警細胞廣播訊息作為案例作為分析對象,以探討其對應情境中的位置描述。為此,首先擴展了道路和災防領域的位置語意之知識本體,以提升語意標註的完整性和適用性。接著,透過實驗結果和實體匹配之評估,驗證本研究方法在這兩個領域中產生位置描述的有效性。本節內容詳細如下:在4.1節中,介紹了實驗區的選擇與圖資;在4.2節說明案例動機、知識本體建構結果,以及語意式位置描述產出;在4.3節呈現案例評估之結果,並討論本研究提出方法適用性。

\section{實驗區及圖資介紹}

本研究選定臺灣地區十萬分之一索引圖中的「9723」圖框範圍(參見圖 12),其介於北緯25度到25.5度和東經121.5度到122度之間,主要考量以下因素:首先,空間物件的多樣性,「9723」區域涵蓋都市、鄉村、山地及海洋水體等多種地理環境,有助於測試位置描述本體在不同空間物件類型下的適用性與語意描述能力,確保模型能夠應對各類地理情境。其次,區域的代表性,「9723」圖框範圍內包含臺灣首都臺北市,擁有密集的道路網絡與豐富的人為設施,使其成為語意式位置描述的理想測試案例,進一步提升研究成果的適用性與擴展性。第三,災害應變與道路通報應用的需求,該區域具有豐富的災害示警歷史記錄,且包含各層級的道路路網,適合作為災害示警與道路分析的應用情境,驗證本研究方法在實際應用中的可行性。最後,空間尺度適中,「9723」圖框範圍既不過於廣泛,以致運算與分析負擔過大,亦不過於狹小,以免影響研究結果的普適性。適當的空間尺度有助於在合理的資源限制下,進行有效的語意標註與分析。

圖資使用為110年瑞竣科技電子地圖集,其圖層清單包含:道路中心線、鐵路、捷運、河流、縣市界、基礎地標、山峰等28類圖層(參見表 9)。

\section{研究結果}

\section{評估與討論}